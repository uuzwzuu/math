\documentclass[11pt, oneside]{article}   	% use "amsart" instead of "article" for AMSLaTeX format

\usepackage{geometry}
 \geometry{
 a4paper,
 total={170mm,257mm},
 left=20mm,
 top=30mm,
 bottom=25mm, 
 }
 
%\usepackage{geometry}                		% See geometry.pdf to learn the layout options. There are lots.
%\geometry{letterpaper}                   		% ... or a4paper or a5paper or ... 
%\geometry{landscape}                		% Activate for rotated page geometry
%\usepackage[parfill]{parskip}    		% Activate to begin paragraphs with an empty line rather than an indent
\usepackage{graphicx}				% Use pdf, png, jpg, or eps§ with pdflatex; use eps in DVI mode
								% TeX will automatically convert eps --> pdf in pdflatex		

\usepackage{amssymb}
\usepackage{amsmath}
\usepackage{amsthm}
\usepackage{fancyhdr}
\usepackage[utf8]{inputenc}
\usepackage[english]{babel}
\usepackage{enumerate}
\usepackage{arcs}
\usepackage{array}
\usepackage{tabularx} 
\usepackage{tikz}
%SetFonts

%SetFonts

\usepackage[inline]{asymptote}


\pagestyle{fancy}
\fancyhf{}
\lhead{Name: \textbf{William Zhong}, Username: \textbf{wzsuperb}, ID: \textbf{40819}}
\rhead{USAMTS, year 34, round 1}
\rfoot{September 30, 2022}


\begin{document}
%\maketitle

\section{Problem 1/1/34}
\vspace{20pt}

\begin{center}
\begin{asy}
size(15cm);
draw((0, 0)--(19, 0));

for(int i=0; i<20; ++i) {
    dot((i, 0)); 
}

draw(arc((7.5, 0), 2.5, 0, 180));
draw(arc((7.5, 0), 0.5, 0, -180));
draw(arc((12.5, 0), 0.5, 0, -180));
draw(arc((6.5, 0), 0.5, 0, 180));
draw(arc((8.5, 0), 0.5, 0, 180));

draw(arc((7.5, 0), 1.5, 0, -180));
draw(arc((12.5, 0), 1.5, 0, -180));
draw(arc((12.5, 0), 2.5, 0, -180));

draw(arc((2.5, 0), 2.5, 0, -180));
draw(arc((7.5, 0), 3.5, 0, 180));
draw(arc((2.5, 0), 1.5, 0, -180));
draw(arc((2.5, 0), 0.5, 0, -180));

draw(arc((15.5, 0), 3.5, 0, 180));
draw(arc((15.5, 0), 2.5, 0, 180));
draw(arc((15.5, 0), 1.5, 0, 180));
draw(arc((15.5, 0), 0.5, 0, 180));

draw(arc((1.5, 0), 1.5, 0, 180));
draw(arc((1.5, 0), 0.5, 0, 180));
draw(arc((17.5, 0), 1.5, 0, -180));
draw(arc((17.5, 0), 0.5, 0, -180));
\end{asy}
\end{center} 

\newpage
\section{Problem 2/1/34}


\newpage
\section{Problem 3/1/34}
We show that, for any positive integer $k$, there is an unique $k$-digit number $A_k$ in base 2022 with the following properties:
\begin{enumerate}
\item All of the digits of $A_k$ (in base 2022) are 1's or 2's, and 
\item $A_k$ is a multiple of the base-10 number $2^{1000}$.
\end{enumerate}

\begin{proof}
When $k=1$ the 1-digit number is $A_1=2_{2022} = 2$, which is a multiple of $2^1$. 

When $k=2$ the 2-digit number is $A_2=12_{2022}=1\cdot2022^1+2=2^1\cdot (1011+1)$, which is a multiple of $2^2$.

For any positive integer $n$, if there is a $n$-digit base-2022 number $A_n=a_n a_{n-1} \cdots a_2 a_1$ ($a_i \in \{1, 2\}, i \in \mathbb{Z^+}$) that is a multiple of base-10 number $2^n$, then we can have another $(n+1)$-digit base-2022 number $A_{n+1}=a_{n+1} a_n \cdots a_2 a_1$ that is a multiple of base-10 number $2^{n+1}$.

Assuming $A_n=r\cdot 2^n, r \in \mathbb{Z^+}$, there are 2 cases to discuss: 
\begin{enumerate}
\item Case 1: $r$ is even. Denoting $r=2\cdot t$, $t\in \mathbb{Z^+}$, then we have $A_n=r\cdot 2^n= 2\cdot t\cdot 2^n= t\cdot 2^{n+1}$. This means $A_n$ is a multiple of $2^{n+1}$. Since $A_{n+1}=a_{n+1}\cdot 2022^n+A_n=a_{n+1}\cdot 2^n\cdot 1011^n + t\cdot 2^{n+1}$, only when $a_{n+1}=2$ should $A_{n+1} = 2^{n+1}\cdot 1011^n + t\cdot 2^{n+1}$ be a multiple of $2^{n+1}$, 

\item Case 2: $r$ is odd. Denoting $r=2t+1$, then we have $A_n=r\cdot 2^n=(2t+1)\cdot 2^n $. Since $A_{n+1}=a_{n+1}\cdot 2022^n+A_n=a_{n+1}\cdot 2^n\cdot 1011^n+(2t+1)\cdot2^n$, only when $a_{n+1}=1$ should $A_{n+1}=2^n\cdot 1011^n+(2t+1)\cdot 2^n=2^n\cdot(1011^n+1)+t\cdot 2^{n+1}$ be a multiple of $2^{n+1}$. Notice that $(1011^n+1)$ is even, and hence $2^n\cdot(1011^n+1)$ is a multiple of $2^{n+1}$.
\end{enumerate} 

The above induction shows that, for any positive integer $k$, we have a $k$-digit number $A_k$ with the properties required. When $k=1000$, we then have $N=A_1000$ that satisfies the properties required.

Given $A_n=r\cdot 2^n$, the parity of $r$ uniquely determines $a_{n+1}$. If $A_n$ is unique, then $A_{n+1}$ is unique. Since $A_1=2$ is unique, by induction we know $A_n$ is unique. Therefore, $N=A_{1000}$ is unique.

\end{proof}

\newpage
\section{Problem 4/1/34}
Winnie should win \$27. 

\begin{proof}
Winnie's strategy is to put the greatest card available to the pile that has the greater sum. Assuming the two piles are pile $A$ and pile $B$, and  Grogg places the first card with number $k$ to pile $A$, the sums of the 2 piles become $k$ and $0$, respectively. Winnie will then place the greatest card available to the same pile $A$ to make the difference larger. To reduce the difference, Grogg has to choose the next greatest card and put it into the other pile $B$. 

Going forward, Winnie chooses the greatest card available and put it into pile $A$, and Grogg puts the next largest card to pile $B$. The game ends when all cards get put into both piles. 

The sum of pile $A$ should always be greater than the sum of pile $B$, since Grogg puts the first card into pile $A$ and in each round afterwards Grogg always puts a smaller card to pile $B$.

To make the difference minimal, Grogg has to choose the smallest card first, e.g. $k=1$. Winnie then chooses  the greatest card with the number 50 and put it into the pile $A$  to make the sum 51. Grogg then chooses the next greatest card with number 49 and put it into pile $B$.

At the end of the game, pile $A$ has all the cards with even numbers and an extra card with number 1. Pile $B$ has all the cards with odd numbers except the card with number 1.

The sum of pile $A$ is $50+48+\cdots +4+2+1=651$, and the sum of pile $B$ is $49+47+\cdots +5+3=624$. The difference is $651-624=27$.

As long as Winnie sticks to her strategy, Grogg has no other strategy to make the difference less than 27.

Therefore, Winnie should win \$27 with the optimal strategy.
\end{proof}





\end{document}  
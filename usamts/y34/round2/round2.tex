\documentclass[11pt, oneside]{article}   	% use "amsart" instead of "article" for AMSLaTeX format

\usepackage{geometry}
 \geometry{
 a4paper,
 total={170mm,257mm},
 left=20mm,
 top=30mm,
 bottom=25mm, 
 }
 
%\usepackage{geometry}                		% See geometry.pdf to learn the layout options. There are lots.
%\geometry{letterpaper}                   		% ... or a4paper or a5paper or ... 
%\geometry{landscape}                		% Activate for rotated page geometry
%\usepackage[parfill]{parskip}    		% Activate to begin paragraphs with an empty line rather than an indent
\usepackage{graphicx}				% Use pdf, png, jpg, or eps§ with pdflatex; use eps in DVI mode
								% TeX will automatically convert eps --> pdf in pdflatex		

\usepackage{amssymb}
\usepackage{amsmath}
\usepackage{amsthm}
\usepackage{fancyhdr}
\usepackage[utf8]{inputenc}
\usepackage[english]{babel}
\usepackage{enumerate}
\usepackage{arcs}
\usepackage{array}
\usepackage{tabularx} 
\usepackage{tikz}
%SetFonts

%SetFonts

\usepackage[inline]{asymptote}


\pagestyle{fancy}
\fancyhf{}
\lhead{Name: \textbf{William Zhong}, Username: \textbf{wzsuperb}, ID: \textbf{40819}}
\rhead{USAMTS, year 34, round 2}
\rfoot{November 20, 2022}


\begin{document}
%\maketitle

\section{Problem 1/2/34}
\vspace{20pt}

\begin{center}
\begin{asy}
size(10cm);


void fillbox(pair a) {
    fill(box(a, a+(1,1)), mediumgray);
}

pair [] p = {(0,0), (2, 2), (2,1), (2,-1), (3,3), (3,2), (3,-3), (4,1), (4,0), (4,-1), (4, -2), (5,1)};

for(int i=0; i<p.length; ++i) {
    fillbox(p[i]);
}

void drawbox(pair a, int x){
    draw(box(a, a+(1,1)));
    label(string(x), a+(0.5,0.5));
}

int [] arr = {16, 24, 6, 17, 25, 12, 3, 8, 18, 23, 11, 5, 13, 4, 9, 19, 14, 15, 7, 20, 23, 2, 1, 10, 21};

int counter = 0;
for(int i=0; i<4; ++i) {
    for(int j=0; j<2*i+1; ++j){
          drawbox((j+3-i, 3-i), arr[counter]);
          ++ counter;
    }
}

for(int i=0; i<3; ++i) {
    for(int j=0; j<2*i+1; ++j){
          drawbox((j+3-i, i-3), arr[counter]);
          ++ counter;
    }
}


\end{asy}
\end{center} 

%\begin{center}
%\includegraphics[width=0.7\textwidth]{imgs/p1.png}
%\end{center}
\newpage
\section{Problem 2/2/34}

There are 3 possible outcomes for Grogg: 
\begin{enumerate}
\item He eats 0 cookies, and the probability is $(1-p)$.
\item He eats 1 cookie, and the probability is $p(1-n p^{n-1}(1-p))$.
\item He eats 2 cookies, and the probability is $n p^n (1-p)$.
 \end{enumerate}
 
 The expectation is
 \begin{align*}
& 0\cdot(1-p) +1\cdot p(1-n p^{n-1}(1-p)) +2\cdot n p^n (1-p)\\
= \quad & p + np^n (1-p).
 \end{align*}
 
 To make the expected value exactly 1, we have
 \begin{align*}
 &p + np^n (1-p)=1\\
 \Rightarrow \quad &np^n(1-p)=1-p, \quad (0<p<1, 1-p \ne 0)\\
 \Rightarrow \quad &np^n=1\\
 \Rightarrow \quad &p=\frac{1}{\sqrt[n]{n}}
 \end{align*}
 
 This is possible for all positive integer $n\ge 2$, where $p=\frac{1}{\sqrt[n]{n}}$. Specifically, when $n=2$, $p=\frac{\sqrt{2}}{2}$.
 
 
 

\end{document}  
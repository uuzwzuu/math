%%%%%%%%%%%%%%%%%%%%%%%%%%%%%%%%%%%%%%%%%%%%%%
%%                                          %%
%% USE THIS FILE TO SUBMIT YOUR SOLUTIONS   %%
%%                                          %%
%% You must have the usamts.tex file in     %%
%% the same directory as this file.         %%
%% You do NOT need to submit this file or   %%
%% usamts.tex with your solutions.  You     %%
%% only need to submit the output PDF file. %%
%%                                          %%
%% DO NOT ALTER THE FILE usamts.tex         %%
%%                                          %%
%% If you have any questions or problems    %%
%% using this file, or with LaTeX in        %%
%% general, please go to the LaTeX          %%
%% forum on the Art Of Problem Solving      %%
%% web site, and post your problem.         %%
%%                                          %%
%%%%%%%%%%%%%%%%%%%%%%%%%%%%%%%%%%%%%%%%%%%%%%

%%%%%%%%%%%%%%%%%%%%%%%%%%%%%%%%%%%%%%%%%%%%
%% DO NOT ALTER THE FOLLOWING LINES
\documentclass[11pt, letterpaper]{article}
\usepackage{amsmath,amssymb,amsthm}
\usepackage[pdftex]{graphicx}
\usepackage{fancyhdr}

\usepackage[inline]{asymptote}
\usepackage{gensymb}


\pagestyle{fancy}
\begin{document}
\include{usamts}
%% DO NOT ALTER THE ABOVE LINES
%%%%%%%%%%%%%%%%%%%%%%%%%%%%%%%%%%%%%%%%%%%%


%% If you would like to use Asymptote within this document (which is optional), 
%% you can find out how at the following URL:
%%
%%   http://www.artofproblemsolving.com/Wiki/index.php/Asymptote:_Advanced_Configuration
%%
%% As explained there, you will want to uncomment the line below.  But be
%% sure to check the website because there are several other steps that must 
%% be followed.
%% \usepackage{asymptote}


%% Enter your real name here
%% Example: \realname{David Patrick}
\realname{William Zhong}

%% Enter your USAMTS username here
%% Example: \username{DPatrick}
%% IMPORTANT
%% If your username contains one of the following characters:
%%      # $ ~ _ ^ % { } &
%% then this character must be preceded by a backslash \
%% for example: if your user name is math_genius, then the line below should be
%% \username{math\_genius}

\username{wzsuperb}

%% Enter your USAMTS ID# here
%% Example: \usamtsid{9999}
\usamtsid{40819}

%% Make sure that the year and round number are correct.
%% Year 36 is the academic year 2024-2025.
\usamtsyear{36}
\usamtsround{1}

%%%%%%%%%%%%%%%%%%%%%%%%%%%%%%%%%%%%%%%%%%%%%%%%%%%%%%%%
%%
%% All solutions go in between the \begin{solution} and
%% the \end{solution} corresponding to the problem
%% number.  
%%
%% For example, suppose Problem 2 is 
%% "Solve for x: x + 3 = 5".
%% Your solution would look like this:
%%
%% \begin{solution}{2}
%% If $x+3=5$, then subtract 3 from both
%% sides of the equation to get $x=5-3=2$, so the 
%% solution is $x=2$.
%% \end{solution}
%%
%%%%%%%%%%%%%%%%%%%%%%%%%%%%%%%%%%%%%%%%%%%%%%%%%%%%%%%%

\begin{solution}{1}
%% Solution to Problem 1 goes here

\begin{center}
\begin{asy}
unitsize(24pt);
pair [] pairs = {(1,0), (1,2), (0,2), (0,4), (1,4), (1,6), (7,6), (7,4), (8,4), (8,2), (7,2), (7,0)};

int n = pairs.length;

for (int i=0; i<n; ++i) 
{
    int k = (i + 1) % n;
    draw(pairs[i] -- pairs[k], linewidth(1pt));
}

for (int i=1; i<6; ++i)
{
    real x1 = 0;
    real x2 = 8; 
    
    if (pairs[i].x > x1) { x1 = pairs[i].x; }
    if (pairs[n-i-1].x < x2) { x2 = pairs[n-i-1].x; }
    
    draw((x1, i) -- (x2, i), dashed);
} 

draw((1,2)--(1,4), dashed);
draw((7,2)--(7,4), dashed);

for (int j=2; j<7; ++j) 
{
    draw((j,0)--(j,6),dashed);
}

int [] nums = {
	0,0,5,0,2,4,2,0,
	0,2,4,2,0,6,5,0,
	6,0,3,1,1,3,2,4,
	3,2,0,1,1,2,0,0,
	0,5,2,0,2,0,3,0,
	0,3,4,6,3,5,4,0
};

for (int i=0; i<6;++i)
for (int j=0; j<8;++j)
{
    int k = i*8 + j;
    if (nums[k] > 0) 
    {
    	label((j+0.5,i+0.5), string(nums[k]));
    }
}

\end{asy}
\end{center}

\end{solution}

\begin{solution}{2}
%% Solution to Problem 2 goes here

\begin{center}
\begin{asy}
unitsize(75pt);
pair o = (0,0);
real r = 1;

draw(circle(o, 1));
real s = sqrt(3)/3; 

pair o1 = (0,s);
pair a = o1 + s*dir(210);
pair b = o1 + s*dir(150);
pair c = o1 + s*dir(90);
pair d = o1 + s*dir(30);
pair e = o1 + s*dir(-30);

draw(o -- a -- b -- c -- d -- e -- cycle);
label("$A$", a, S);
label("$B$", b, N);
label("$C$", c, N);
label("$D$", d, N);
label("$E$", e, S);
label("$O$", o, S);

draw(o -- c, blue);
draw(b -- e, blue);
draw(a -- d, blue);

draw(o -- b, blue);

label("$K$", o1, SE);



\end{asy}
\end{center}

As illustrated above, we have a unit circle $O$ and a regular hexagon $ABCDEO$ together satisfying the conditions of the problem.

Suppose point $K$ is the center of the regular hexagon $ABCDEO$, connecting the 6 vertices with $K$ divides the hexagon into 6 congruent equilaterals. 

To show that $\triangle AKB$ is an equilateral: $BK = AK$ and $\angle BKA = \frac{360\degree}{6} = 60\degree$. Similarly, we can show the other 5 triangles are also equilaterals.

Since $\triangle ABO$ is an isosceles triangle ($AO = AB$) and $AK$ bisects $\angle OAB$ ($\angle OAK = \angle BAK = 60\degree $), we have that $AK \perp BO$ and $AK$ bisects $BO$. Therefore, the area of $\triangle ABK$ 
$$ \frac{1}{2}\cdot AK \cdot \frac{BO}{2} = \frac{1}{2}\cdot \frac{1}{2\cdot \cos 60\degree}\cdot \frac{1}{2} =\frac{\sqrt{3}}{12},$$

and the area of the hexagon is 

$$6\cdot \frac{\sqrt{3}}{12} = \frac{\sqrt{3}}{2}.$$
\end{solution}

\begin{solution}{3}
%% Solution to Problem 3 goes here
\end{solution}

\begin{solution}{4}
%% Solution to Problem 4 goes here
\end{solution}

\begin{solution}{5}
%% Solution to Problem 5 goes here
\end{solution}

%%
%% DO NOT ALTER THE FOLLOWING COMMAND
\end{document}
%% DO NOT ALTER THE PREVIOUS COMMAND
%%

%%%%%%%%%%%%%%%%%%%%%%%%%%%%%%%%%%%%%%%%%%%%%%
%%                                          %%
%% You may want to run LaTeX on this file   %%
%% to make sure that it compiles correctly  %%
%% before you submit it.                    %%
%%                                          %%
%% If you run LaTeX on this file before     %%
%% entering any solutions, the output file  %%
%% will be blank.  This is normal.          %%
%%                                          %%
%% Make sure that the file usamts.tex is    %%
%% in the same directory as this file.      %%
%% If the compiler hangs or asks for        %%
%% fancyhdr.sty when you compile, then go   %%
%% to the My Forms portion of My USAMTS     %%
%% at www.usamts.org to download the        %%
%% fancyhdr.sty file.                       %%
%%                                          %%
%% Questions?  Problems?  Go to the LaTeX   %%
%% forum on www.artofproblemsolving.com     %%
%%                                          %%
%%%%%%%%%%%%%%%%%%%%%%%%%%%%%%%%%%%%%%%%%%%%%%
